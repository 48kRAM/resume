%%%%%%%%%%%%%%%%%%%%%%%%%%%%%%%%%%%%%%%
% Resume of Joshua B. Malone 
% 
% This document uses the Deedy CV/Resume XeLaTeX Template
% and must be compiled with XeLaTeX
%
%%%%%%%%%%%%%%%%%%%%%%%%%%%%%%%%%%%%%%%%

\documentclass[letterpaper]{deedy-resume}
\usepackage[none]{hyphenat}
\begin{document}

%----------------------------------------------------------------------
%       TITLE SECTION
%----------------------------------------------------------------------

\namesection{Joshua Malone}{
josh.malone@gmail.com | https://github.com/48kRAM | 434-234-4716
}

%------------------
%     LEFT COLUMN
%------------------

\begin{minipage}[t]{0.27\textwidth}
\section{Languages}
Bash \textbullet{} PHP \textbullet{} Perl \textbullet{} HTML/CSS \\
Python \textbullet{} JavaScript \textbullet{} C \\
SQL \textbullet{} MarkDown

\sectionspace
\section{Software}
BIND \textbullet{} Samba \textbullet{} dovecot \\
ISC-DHCPd \textbullet{} Apache HTTPd \\
Sendmail \textbullet{} LDAP \textbullet{} git \\
GitLab \textbullet{}  MatterMost \\
Openfire \textbullet{} GNU Mailman \\
Nagios \textbullet{} BackupPC \textbullet{} P5 \\
Nextcloud \textbullet{} Munki \textbullet{} OpenSSL \\
Procmail \textbullet{} MailScanner

\sectionspace

\section{Education}
\subsection{Virginia Polytechnic}
\subsection{Institute}
\location{Blacksburg, VA}
\descript{Computer Engineering}

\sectionspace

\section{Coursework}
Computer Architecture \\
TCP/IP Network\\
\hspace{3pt} Application Development \\
Microprocessors \\
Digital Design

\sectionspace

\section{Interests}
\vspace{\topsep} % Hacky fix for awkward extra vertical space

\begin{tightitemize}
	\item Vintage computers (1975-1996)
	\item Vintage gaming
	\item Former amateur radio operator
	\item Data and network security
\end{tightitemize}

\end{minipage} % End of left column
\hfill
%
%------------------------------------------------------
%       RIGHT COLUMN
%------------------------------------------------------
\begin{minipage}[t]{0.69\textwidth} % Right column
\vspace{3pt}
\summary{Experienced Unix systems administrator (especially Linux and MacOS) with an emphasis on automation, repeatability and security.} 

\section{Experience}
	
\subsection{National Radio Astronomy Observatory}
\descript{Linux and Mac Systems Administrator}
\location{Charlottesville, VA | 2005 - Present}

\begin{tightitemize}
	\item Administered Red Hat Enterprise servers, ver. 5, 6, and 7
	\item Critical infrastructure servers, incl. web, email, DNS, DHCP, backup, monitoring, and file services
	\item Observatory lead for Mac platform from 2009 to 2015
	\item Lead observatory-wide deployment of NRAO's first central management system for Mac OSX (Munki)
	\item Lead observatory-wide deployment of Nagios for automated service monitoring and trending
	\item Lead observatory-wide deployment of Gitlab SCM, including user-facing	documentation and training
	\item Utilized perl and net-LDAP to integrate Oracle J.D. Edwards with Active Directory, NIS, and Mailman to automate AD group membership, rfc-2307 AD unix attributes, and mailing list membership
	\item Managed multiple MySQL instances, including replication to off-site disaster recovery systems
	\item Lead observatory-wide rollout of BackupPC software to provide  server backups and reduce commercial backup software license costs for the observatory
	\item Developed numerous in-house web applications, including an online phone directory, account consistency checker (for helpdesk staff), and helpdesk ticketing system
	\item Administered NetApp filers for NFS, CIFS, and iSCSI running Data OnTap 7 and Clustered Data OnTap
	\item Designed and implemented upgrades to audiovisual presentation and conference systems in meeting facilities
	\item Wrote extensive documentation for both end-users and IT staff
	\item Maintained detailed logs of production server changes
	\item Three-time recipient of observatory "Star Award" for significant achievements
\end{tightitemize}

\vspace{6pt}
\subsection{Applied Data Systems (now Eurotech)}
\descript{Linux Engineer}
\vspace{2pt}
\location{Columbia, MD (Charlottesville, VA office) | Mar. 2004 - Sept. 2005}

\begin{tightitemize}
	\item Packaged and maintained Debian GNU/Linux distributions for ARM-based single-board embedded systems
	\item Wrote and maintained documentation for internal and customer use under ISO-9001 document control
	\item Ported or wrote drivers for CAN controllers, video decoders, and video accelerators to run on embedded systems
	\item Assisted with introduction of the Debian GNU/Linux operating system as a	standard customer option for ADS embedded systems
	\item Wrote board imaging tools for use by manufacturing technicians
\end{tightitemize}

\end{minipage}
\newpage

\section{Technical Presentations}
\begin{tightitemize}
\vspace{8pt}
\item \textit{Git for Collabirative Development} - Multiple internal presentations to observatory staff
\item Automated System Monitoring - Fall 2015 LSP Conference, University of Virginia \\
\item Advanced systems monitoring with Nagios, PNP and Nconf - 2015 Mac Admins Conference at Penn State \\
\item What the heck is wrong with my server?!? - 2014 Mac Admins Conference at Penn State \\
\item Managing OSX Systems with Munki - Fall 2012 LSP Conference, University of
Virginia \\
\end{tightitemize}

\sectionspace
\section{Open Source Code \& Contributions}
\vspace{8pt}
\begin{tightitemize}
	\item "JSTimer" - A graphical session timer for scientific talks with automatic Q\&A rollover.
	\item Contributions to "Webmin" project to enhance DHCP module
	
\end{tightitemize}

\end{document}